\documentclass[preview]{standalone}
\usepackage[utf8]{inputenc} 
\usepackage[russian]{babel}

\textheight=24cm % высота текста
\textwidth=16cm % ширина текста
\oddsidemargin=0pt % отступ от левого края
\topmargin=-1.5cm % отступ от верхнего края
\parindent=24pt % абзацный отступ
\parskip=5pt % интервал между абзацами
\tolerance=2000 % терпимость к "жидким" строкам
\flushbottom % выравнивание высоты страниц
\usepackage{indentfirst}

\usepackage{arydshln}
\usepackage[fleqn]{amsmath}
\usepackage{amssymb}
\usepackage[T1]{fontenc}
\usepackage{mathtools}
\usepackage{color}
\usepackage{ulem}
\normalem
\makeatletter
\newenvironment{sqcases}{\matrix@check\sqcases\env@sqcases}{\endarray\right.}\def\env@sqcases{\let\@ifnextchar\new@ifnextchar\left\lbrack\def\arraystretch{1.2}\array{@{}l@{\quad}l@{}}}\makeatother

% объявляем новую команду для переноса строки внутри ячейки таблицы
\newcommand{\specialcell}[2][c]{\begin{tabular}[#1]{@{}c@{}}#2\end{tabular}}

	\title{Math formulae.}

\begin{document} \fontsize{12pt}{16pt}\selectfont
%--------------------------------------------------------------------------------%
%--------------------------------------------------------------------------------%

\maketitle

\tableofcontents

\thispagestyle{empty}

\newpage

\setcounter{page}{1}

%--------------------------------------------------------------------------------%
%--------------------------------------------------------------------------------%

\input{formulae}

%--------------------------------------------------------------------------------%
%--------------------------------------------------------------------------------%
\end{document}