\section{Всякое по мелочам}

$ |a| = \left[ \begin{aligned}
	& a,\text{если\ } a\ge 0 \\
	& -a,\text{если\ } a < 0
\end{aligned}\right. $

\begin{center}
	Уравнение касательной к графику функции $\displaystyle y = f(x) $ в точке $\displaystyle M_0(x_0, f(x_0)) $ имеет вид:
	$ y = f'(x_0)\, x + (f(x_0) - f'(x_0)\, x_0) $
\end{center}

$ a^4 + a^2 + 1 = (a^2 - a + 1)\, (a^2 - a + 1) $

$ a^2 + b^2 = (a + b)^2 - 2\, a\, b $

$ (a \pm b)^3 = a^3\pm b^3\pm 3\, a\, b\, (a\pm b) $

$ \sqrt{\mathstrut a\pm \sqrt{\mathstrut b}} = \sqrt{\mathstrut \frac{a + \sqrt{\mathstrut a^2 - b}}{2}}\pm \sqrt{\mathstrut \frac{a - \sqrt{\mathstrut a^2 - b}}{2}} $

\begin{center}
	Квадратное уравнение:
	$ a\, x^2 + b\, x + c = 0 \Rightarrow $ $ \left[ \begin{aligned}
		& a\, (x - x_1)\, (x - x_2) = 0 \\
		& D = b^2 - 4\, a\, c \\
		& \left\{ \begin{aligned}
			& D > 0 \\
			& x_{1/2} = \frac{-b\pm \sqrt{\mathstrut D}}{2\, a}
		\end{aligned}\right. \\
		& \left\{ \begin{aligned}
			& D = 0 \\
			& x_{1/2} = \frac{-b}{2\, a}
		\end{aligned}\right.
	\end{aligned}\right. $
\end{center}

%--------------------------------------------------------------------------------%

\section{Свойства степени}

$ a^0 = 1 $

$ a^m = \underbrace{a\, a \, \ldots \, a }_{m сомножителей} $

$ a^{-m} = \frac{1}{a^m} $

$ a^n\, a^m = a^{n+m} $

$ \frac{a^n}{a^m} = a^{n-m} $

$ (a^n)^m = a^{n\, m} $

$ (a\, b)^n = a^n\, a^m $

$ \left(\frac{a}{b} \right)^n = \frac{a^n}{b^n} $

%--------------------------------------------------------------------------------%

\section{Свойства корня}

$ \sqrt[n]{\mathstrut a\, b} = \sqrt[n]{\mathstrut a} \, \sqrt[n]{\mathstrut b} $

$ \sqrt[n]{\mathstrut \frac{a}{b}} = \frac{\sqrt[n]{\mathstrut a}}{\sqrt[n]{\mathstrut b}} $

$ (\sqrt[n]{\mathstrut a})^k = \sqrt[n]{\mathstrut a^k} = a^{\frac{k}{n}} $

$ \sqrt[2n + 1]{\mathstrut -a} = -\sqrt[2n + 1]{\mathstrut a} $

$ \sqrt[n]{\mathstrut \sqrt[k]{\mathstrut a}} = \sqrt[n\, k]{\mathstrut a} $

$ \sqrt[n]{\mathstrut a} = \sqrt[n\, k]{\mathstrut a^k} $

$ \sqrt[2n]{\mathstrut a^{2n}} = |a| $

%--------------------------------------------------------------------------------%

\section{Свойства логарифма}

\begin{center}
	если $ a^x = b $, то $ \log_a b = x $.
\end{center}

$ a^{\log_b c} = c^{\log_b a} $

$ a^{\log_a x} = x $

$ \log_a (x\, y) = \log_a |x| + \log_a |y| $

$ \log_a \frac{x}{y} = \log_a |x| - \log_a |y| $

$ \log_a x^{2\, m} = 2m \log_a |x| $

$ \log_a b = \frac{\log_x b}{\log_x a} $

$ \log_a x = \frac{1}{\log_b a} $

$ \log_{a^k} x^m = \frac{m}{k} \log_a x $

$ \log_a 1 = 0 $

$ \log_a a = 1 $

$ \log_{\phi(x)} f(x) \ge \log_{\phi(x)} g(x) \Rightarrow $ $ \left[
	\begin{aligned}
		\left\{ \begin{aligned}
			& f(x) \ge g(x) \\
			& g(x) > 0 \\
			& \phi(x) > 1
		\end{aligned} \right.
		\left\{ \begin{aligned}
			& f(x) \le g(x) \\
			& f(x) > 0 \\
			& 0 < \phi(x) < 1
		\end{aligned} \right.
	\end{aligned} \right. $

$ \log_{\phi(x)} f(x) > \log_{\phi(x)} g(x) \Rightarrow $ $ \left[
	\begin{aligned}
		\left\{ \begin{aligned}
			& f(x) > g(x) \\
			& g(x) > 0 \\
			& \phi(x) > 1
		\end{aligned} \right.
		\left\{ \begin{aligned}
			& f(x) < g(x) \\
			& f(x) > 0 \\
			& 0 < \phi(x) < 1
		\end{aligned} \right.
	\end{aligned} \right. $

%--------------------------------------------------------------------------------%
	
\section{Комбинаторика}

%--------------------------------------------------------------------------------%
	
\section{Прогрессии}

\subsection{Арифметическая}

\subsection{Геометрическая}

%--------------------------------------------------------------------------------%

\section{Тригонометрия}

\subsection{Графики}

\subsection{Основное}

$ \sin^2 \alpha + \cos^2 \alpha = 1 $

$ \tg^2 \alpha + 1 = \frac{1}{\cos^2 \alpha} $

$ \ctg^2 \alpha + 1 = \frac{1}{\sin^2 \alpha} $

$ \tg \alpha\, \ctg \alpha = 1 $

$ \tg \alpha = \frac{\sin \alpha}{\cos \alpha} $

\subsection{Суммы углов}

$ \sin(\alpha\pm \beta) = \sin \alpha\, \cos \beta \pm \cos \alpha\, \sin \beta $

$ \cos(\alpha\pm \beta) = \cos \alpha\, \cos \beta \mp \sin \alpha\, \sin \beta $

$ \tg (\alpha \pm \beta) = \frac{\tg \alpha \pm \tg \beta}{1 \mp \tg \alpha \, \tg \beta} $

\subsection{Двойные и тройные углы}

$ \sin 2\, \alpha = 2\, \sin \alpha\, \cos \alpha $

$ \cos 2\, \alpha = \cos^2 \alpha - \sin^2 \alpha = $ $ 2\, \cos^2 \alpha - 1 = 1 - 2\, \sin^2 \alpha $

$ \tg 2\alpha = \frac{2\, \tg \alpha}{1 - \tg^2 \alpha} = \frac{2}{\ctg \alpha - \tg \alpha} $

$ \cos 3\, \alpha = 4\, \cos^3 \alpha - 3\, \cos \alpha $

$ \sin 3\, \alpha = 3\, \sin \alpha - 4\, \sin^3 \alpha $

$ \tg 3\, \alpha = \frac{3\, \tg \alpha - \tg^3 \alpha}{1 - 3\, \tg^2 \alpha} $

\subsection{Сумма функций}

$ \sin \alpha \pm \sin \beta = 2\, \sin \frac{\alpha \pm \beta}{2}\, \cos \frac{\alpha \mp \beta}{2} $

$ \cos \alpha \pm \cos \beta = 2\, \cos \frac{\alpha \pm \beta}{2}\, \cos \frac{\alpha \mp \beta}{2} $

$ a\, \sin \alpha + b\, \cos \alpha = \sqrt{\mathstrut a^2 + b^2}\, \sin \left(\alpha + \arcsin \frac{a}{\sqrt{\mathstrut a^2 + b^2}}\right) $

$ \tg \alpha \pm \tg \beta = \frac{\sin (\alpha \pm \beta)}{\cos \alpha\, \cos \beta} $

$ \ctg \alpha \pm \ctg \beta = \frac{\pm\sin (\alpha \pm \beta)}{\sin \alpha\, \sin \beta} $

$ \ctg \alpha \pm \tg \beta = \frac{\cos (\alpha \mp \beta)}{\sin \alpha\, \cos \beta} $

\subsection{Произведение функций}

$ \sin \alpha\, \cos \beta = \frac{1}{2}\, (\sin (\alpha - \beta) + \sin (\alpha + \beta)) $

$ \sin \alpha\, \sin \beta = \frac{1}{2}\, (\cos (\alpha - \beta) - \cos (\alpha + \beta)) $

$ \cos \alpha\, \cos \beta = \frac{1}{2}\, (\cos (\alpha - \beta) + \cos (\alpha + \beta)) $

\subsection{Понижение степени и половинный угол}

$ \cos^2 \alpha = \frac{1 + \cos 2\, \alpha}{2} $

$ \sin^2 \alpha = \frac{1 - \cos 2\, \alpha}{2} $

$ \sin^3 \alpha = \frac14(3\sin \alpha - \sin 3 \alpha) $

$ \cos^3 \alpha = \frac14(3\cos \alpha + \cos 3 \alpha) $

$ \tg \frac{\alpha}{2} = \pm \sqrt{\mathstrut \frac{1 - \cos \alpha}{1 + \cos \alpha}} = \frac{\sin \alpha}{1 + \cos \alpha} $

\subsection{Универсальная тригонометрическая подстановка}

{\renewcommand{\arraystretch}{2}
\begin{tabular}[t]{||c|c||}
	\hline
		$ \tg \cfrac{\alpha}{2} = t $ & 		$ \sin \alpha = \cfrac{2t}{1 + t^2} $ \tabularnewline
	\hline
		$ \alpha = 2\arctg t $ & 		$ \cos \alpha = \cfrac{1 - t^2}{1 + t^2} $ \tabularnewline
	\hline
		$ d\alpha = \cfrac{2dt}{1+t^2}$ & 	$ \tg \alpha = \cfrac{2t}{1-t^2} $ \tabularnewline
	\hline
\end{tabular}}

\subsection{Таблица с суммой углов}

\begin{tabular}[t]{||c|c|c|c|c|c|c|c|c||}
	\hline
		$ \alpha $ &
			\specialcell{$ \frac{\pi}{2} - \alpha $ \\ $ 90^{\circ} - \alpha $} &
			\specialcell{$ \frac{\pi}{2} + \alpha $ \\ $ 90^{\circ} + \alpha $} &
			\specialcell{$ \pi - \alpha $ \\ $ 180^{\circ} - \alpha $} &
			\specialcell{$ \pi + \alpha $ \\ $ 180^{\circ} + \alpha $} &
			\specialcell{$ \frac{3\, \pi}{2} - \alpha $ \\ $ 270^{\circ} - \alpha $} &
			\specialcell{$ \frac{3\, \pi}{2} + \alpha $ \\ $ 270^{\circ} + \alpha $} &
			\specialcell{$ 2\, \pi - \alpha $ \\ $ 360^{\circ} - \alpha $} &
			\specialcell{$ 2\, \pi + \alpha $ \\ $ 360^{\circ} + \alpha $} \tabularnewline
	\hline
		$ \sin \alpha $ & 	$ \cos \alpha $ & 	$ \cos \alpha $ & 	$ \sin \alpha $ & 	$ -\sin \alpha $ & 	$ -\cos \alpha $ & 	$ -\cos \alpha $ & 	$ -\sin \alpha $ & 	$ \sin \alpha $ \tabularnewline
	\hline
		$ \cos \alpha $ & 	$ \sin \alpha $ & 	$ -\sin \alpha $ & 	$ -\cos \alpha $ & 	$ -\cos \alpha $ & 	$ -\sin \alpha $ & 	$ \sin \alpha $ & 	$ \cos \alpha $ & 	$ \cos \alpha $ \tabularnewline
	\hline
		$ \tg \alpha $ & 	$ \ctg \alpha $ & 	$ -\ctg \alpha $ & 	$ -\tg \alpha $ & 	$ \tg \alpha $ & 	$ \ctg \alpha $ & 	$ -\ctg \alpha $ & 	$ -\tg \alpha $ & 	$ \tg \alpha $ \tabularnewline
	\hline
		$ \ctg \alpha $ & 	$ \tg \alpha $ & 	$ -\tg \alpha $ & 	$ -\ctg \alpha $ & 	$ \ctg \alpha $ & 	$ \tg \alpha $ & 	$ -\tg \alpha $ & 	$ -\ctg \alpha $ & 	$ \ctg \alpha $ \tabularnewline
	\hline
\end{tabular}

\subsection{Тригонометрические уравнения}

\begin{tabular}[t]{||c|c|c||}
	\hline
		Уравнение & Решение & Условие \tabularnewline
	\hline
		$ \sin x = a $ & 	$ x = (-1)^k\, \arcsin a + \pi\, k $ & 	$ |a| \le 1 $ \tabularnewline
	\hline
		$ \cos x = a $ & 	$ x = \pm \arccos a + 2\, \pi\, k $ & 	$ |a| \le 1 $ \tabularnewline
	\hline
		$ \tg x = a $ & 	$ x = \arctg a + \pi\, k $ & 	$ - $ \tabularnewline
	\hline
		$ \ctg x = a $ & 	$ x = \arcctg a + \pi\, k $ & 	$ - $ \tabularnewline
	\hline
\end{tabular}

Частные случаи:

\begin{tabular}[t]{||c|c||c|c||}
	\hline
		Уравнение & Решение & Уравнение & Решение \tabularnewline
	\hline
		$ \sin x = 0 $ & 	$ x = \pi\, k $ & 				$ \cos x = 0 $ &	$ x = \frac{\pi}{2} + \pi\, k $ \tabularnewline
	\hline
		$ \sin x = 1 $ & 	$ x = \frac{\pi}{2} + 2\, \pi\, k $ & 	$ \cos x = 1 $ & 	$ x = 2\, \pi\, k $ \tabularnewline
	\hline
		$ \sin x = -1 $ & 	$ x = -\frac{\pi}{2} + 2\, \pi\, k $ & 	$ \cos x = -1 $ & 	$ x = \pi + 2\, \pi\, k $ \tabularnewline
	\hline
		$ \tg x = 0 $ & 	$ x = \pi\, k $ & 				$ \ctg x = 0 $ & 	$ x = \frac{\pi}{2} + \pi\, k $ \tabularnewline
	\hline
		$ \tg x = 1 $ & 	$ x = \frac{\pi}{4} + \pi\, k $ & 		$ \ctg x = 1 $ & 	$ x = \frac{\pi}{4} + \pi\, k $ \tabularnewline
	\hline
		$ \tg x = -1 $ & 	$ x = -\frac{\pi}{4} + \pi\, k $ & 		$ \ctg x = -1 $ & 	$ x = \frac{3\, \pi}{4} + \pi\, k $ \tabularnewline
	\hline
\end{tabular}

\subsection{Обратные тригонометрические функции:}

{\renewcommand{\arraystretch}{1}
\begin{tabular}[t]{||c||c||c||}
	\hline
		$ \sin\arcsin x = x $            & $ \sin\arccos x = \sqrt{1-x^2} $          & $ \arcsin(-x) = -\arcsin x $              \tabularnewline
	\hline
		$ \arcsin\sin\alpha = \alpha $   & $ \cos\arcsin x = \sqrt{1-x^2} $            & $ \arccos(-x) = \pi - \arccos x $         \tabularnewline
	\hline
	\hline
		$ \cos\arccos x = x $            & $ \sin\arctg x = \cfrac{x}{\sqrt{1+x^2}} $ & $ \arctg(-x) = -\arctg x $                \tabularnewline
	\hline
		$ \arccos\cos\alpha = \alpha $   & $ \cos\arctg x = \cfrac{1}{\sqrt{1+x^2}} $ & $ \arcctg(-x) = \pi - \arcctg x $         \tabularnewline
	\hline
	\hline
		$ \tg\arctg x = x $              & $ \tg\arcsin x = \cfrac{x}{\sqrt{1-x^2}} $ & $ \arcsin x + \arccos x = \cfrac{\pi}{2} $ \tabularnewline
	\hline
		$ \arctg\tg\alpha = \alpha $     & $ \tg\arccos x = \cfrac{\sqrt{1-x^2}}{x} $ & $ \arctg x + \arcctg x = \cfrac{\pi}{2} $  \tabularnewline
	\hline
\end{tabular}}

\subsection{Сумма обратных тригонометрических функций}

{\renewcommand{\arraystretch}{1.5}
\begin{tabular}[t]{||c||}
	\hline
		$ \arcsin x - \arcsin y =  \arcsin x + \arcsin(-y) $ \\
		$ \Sigma_{\sin{}} = \arcsin\left(x\sqrt{\mathstrut 1-y^2}+y\sqrt{\mathstrut 1-x^2}\right) $ \\
	\hline
		$ \boxed{\arcsin x + \arcsin y =} $ \\
		$ \left\{ \begin{aligned}
			\Sigma_{\sin{}}, \quad & xy \leqslant 0,\quad & x^2 + y^2 \leqslant 1 \\
			\pi-\Sigma_{\sin{}}, \quad & x > 0, y > 0,\quad  & x^2 + y^2 > 1 \\
			-\pi-\Sigma_{\sin{}}, \quad & x < 0, y < 0,\quad & x^2 + y^2 > 1 \\
		\end{aligned} \right. $ \\
	\hline
	\hline
		$ \arccos x - \arccos y =  -\pi + \arccos x + \arccos(-y) $ \\
		$ \Sigma_{\cos{}} = \arccos\left(xy+\sqrt{\mathstrut 1-x^2}\sqrt{\mathstrut 1-y^2}\right) $ \\
	\hline
		$ \boxed{\arccos x + \arccos y =} $
		$ \left\{ \begin{aligned}
			\Sigma_{\cos{}}, \quad & x \geqslant -y \\
			2\pi-\Sigma_{\cos{}}, \quad & x < -y
		\end{aligned} \right. $ \\
	\hline
	\hline
		$ \arctg x - \arctg y =  \arctg x + \arctg(-y) $ \\
		$ \Sigma_{\tg{}} = \arctg\cfrac{x+y}{1-xy} $ \\
	\hline
		$ \boxed{\arctg x + \arctg y =} $
		$ \left\{ \begin{aligned}
			\Sigma_{\tg{}}, \quad & & xy < 1 \\
			\pi+\Sigma_{\tg{}}, \quad & x > 0, \quad & xy>1 \\
			-\pi+\Sigma_{\tg{}}, \quad & x < 0, \quad & xy>1 \\
		\end{aligned} \right. $ \\
	\hline
\end{tabular}}

\subsection{Таблица значений функций}

\subsubsection{Для стандартных углов}

\subsubsection{Для особых углов}

\subsubsection{Для нестандартных углов}

%--------------------------------------------------------------------------------%

\section{Гиперболические функции}

\subsection{Графики}

\subsection{Основные}

$$ \ch^2 x - \sh^2 x = 1 $$

$$ \ctg^2 x - 1 = \frac{1}{\sh^2 x} $$

$$ 1 - \th^2 x = \frac{1}{\ch^2 x} $$

$$ \th x \, \cth x = 1 $$

$$ \th x = \frac{\sh x}{\ch x} $$

\subsection{Суммы углов}

$$ \sh(x \pm y) = \sh x \, \ch y \pm \sh y \, \ch x $$

$$ \ch(x \pm y) = \ch x \, \ch y \pm \sh y \, \sh x $$

$$ \th(x \pm y) = \frac{\th x \pm \th y}{1 \pm \th x \, \th y} $$

\subsection{Двойные и тройные углы}

$$ \sh 2x = 2\sh x \, \ch x $$

$$ \ch 2x = \ch^2 x + \sh^2 x = 2\ch^2 x - 1 = 1 + 2\sh^2 x $$

$$ \th 2x = \frac{2\th x}{1 + \th^2 x} = \frac{2}{\th x + \cth x} $$

$$ \sh 3x = 4\sh^3 x + 3\sh x $$

$$ \ch 3x = 4\ch^3 x - 3\ch x $$

$$ \th 3x = \th x \cdot \frac{3+\th^2 x}{1 + 3\th^2 x} $$

\subsection{Сумма функций}

$$ \sh x \pm \sh y = 2\sh\frac{x\pm y}{2}\,\ch\frac{x\mp y}{2} $$

$$ \ch x + \ch y = 2\ch\frac{x+y}{2}\,\ch\frac{x-y}{2} $$

$$ \ch x - \ch y = 2\sh\frac{x+ y}{2}\,\sh\frac{x- y}{2} $$

$$ \th x \pm \th y = 2\frac{\sh(x\pm y)}{\ch x\, \ch y} $$

\subsection{Произведение функций}

$$ \sh x \, \sh y = \frac12(\ch(x+y)-\ch(x-y)) $$

$$ \sh x \, \ch y = \frac12(\sh(x+y)+\sh(x-y)) $$

$$ \ch x \, \ch y = \frac12(\ch(x+y)+\ch(x-y)) $$

\subsection{Понижение степени и половинный угол}

$$ \ch^2 x = \frac{\ch 2x+1}{2} $$

$$ \th^2 x = \frac{\ch 2x-1}{2} $$ 

\subsection{Универсальная гиперболическая подстановка}



\subsection{Обратные гиперболические функции}

\subsection{Сумма обратных гиперболических функций}

%--------------------------------------------------------------------------------%

\section{Математический анализ}

\subsection{Пределы}

\subsection{Ряды}

\subsection{Дифференциирование}

\subsection{Интегрирование}

\subsubsection{Свойства}

\subsubsection{Таблица неопределенных интегралов}

\subsection{Исследование функции}

\subsection{Функции нескольких переменных}

%--------------------------------------------------------------------------------%

\section{Линейная алгебра}

\section{Матрицы}

\subsection{Вектора линейного пространства}

Обозначение базиса: $\mathbb{e} = (\boldsymbol{e_1}, \boldsymbol{e_2}, \dots, \boldsymbol{e_n})$.

Обозначение координатного столбца вектора в базисе $\mathbb{e}$: $[x]_\mathbb{e} = 
\begin{pmatrix}
	x_1\\
	x_2\\
	\dots\\
	x_n
\end{pmatrix} $  

Любой вектор можно выразить через базис и координатный столбец: 
$$
	\boldsymbol{x} = \mathbb{e}\cdot [x]_\mathbb{e} = (\boldsymbol{e_1}, \boldsymbol{e_2}, \dots, \boldsymbol{e_n}) \cdot 
	\begin{pmatrix}
		x_1\\
		x_2\\
		\dots\\
		x_n
	\end{pmatrix} = x_1\boldsymbol{e_1} + x_2\boldsymbol{e_2} + \dots + x_n\boldsymbol{e_n}; 
$$

Линейные операции над вектором равносильны линейным операциям над его координатным столбцом:

$$ [c\cdot \boldsymbol{x}]_\mathbb{e} = c\cdot [\boldsymbol{x}]_\mathbb{e} $$
$$ [\boldsymbol{x} + \boldsymbol{y}]_\mathbb{e} = [\boldsymbol{x}]_\mathbb{e} + [\boldsymbol{y}]_\mathbb{e} $$

\section{Получение матрицы перехода к новому базису}

Если вектора базиса $\mathbb{a}$ заданы через вектора $\mathbb{e}$, т.е.

\begin{equation*} 
	\begin{cases}
		\boldsymbol{a_1} = \alpha_{1,1}\boldsymbol{e_1} + \alpha_{1,2}\boldsymbol{e_2} + \dots + \alpha_{1,n}\boldsymbol{e_n} = \mathbb{e} \cdot [a_1]_\mathbb{e} \\
		\boldsymbol{a_2} = \alpha_{2,1}\boldsymbol{e_1} + \alpha_{2,2}\boldsymbol{e_2} + \dots + \alpha_{2,n}\boldsymbol{e_n} = \mathbb{e} \cdot [a_2]_\mathbb{e} \\
		\dots \\
		\boldsymbol{a_n} = \alpha_{n,1}\boldsymbol{e_1} + \alpha_{n,2}\boldsymbol{e_2} + \dots + \alpha_{n,n}\boldsymbol{e_n} = \mathbb{e} \cdot [a_n]_\mathbb{e}
	\end{cases}
	\Leftrightarrow
	\begin{matrix}
		P_\mathbb{ea}^T \cdot \mathbb{e}^T& =& \mathbb{a}^T \\
		\mathbb{e} \cdot P_\mathbb{ea}& =& \mathbb{a}\\
		\mathbb{a} \cdot P_\mathbb{ea}^{-1}& =& \mathbb{e}\\
		\mathbb{a} \cdot P_\mathbb{ae}& =& \mathbb{e}\\
	\end{matrix}
\end{equation*} 

То матрица перехода $P_\mathbb{ea}$ будет равна:

\[
P_\mathbb{ea} = 
\left([a_1]_\mathbb{e}, [a_2]_\mathbb{e}, \dots, [a_n]_\mathbb{e}\right) = 
\begin{pmatrix}
	\alpha_{1,1}& \alpha_{2, 1}& \dots& \alpha_{n,1}\\
	\alpha_{1,2}& \alpha_{2, 2}& \dots& \alpha_{n,2}\\
	\dots& \dots& \dots& \dots\\
	\alpha_{1,n}& \alpha_{2, n}& \dots& \alpha_{n,n}
\end{pmatrix}
\]

\section{Формулы перехода к новому базису}

$$ P_\mathbb{ea} = P_\mathbb{ae}^{-1} $$
$$ \mathbb{a} = \mathbb{e} \cdot P_\mathbb{ea} $$
$$ P_\mathbb{ad} = P_\mathbb{ae}\cdot P_\mathbb{ed} $$
$$ [x]_\mathbb{a} = P_\mathbb{ea} \cdot [x]_\mathbb{e} $$

\section{Линейный оператор}

$A$ - линейный оператор.

$A_\mathbb{e}$ - матрица $A$ в базисе $\mathbb{e}$.

$c$ - некоторое число.
 
$$ A(\boldsymbol{x}) = y $$
$$ A(\boldsymbol{x_1} + \boldsymbol{x_2}) = A(\boldsymbol{x_1}) + A(\boldsymbol{x_2}) $$
$$ A(c\cdot\boldsymbol{x}) = c\cdot A(\boldsymbol{x}) $$
$$ [y]_\mathbb{e} = A_\mathbb{e} \cdot [x]_\mathbb{e} $$
$$ A_\mathbb{a} = P_\mathbb{ea}^{-1} \cdot A_\mathbb{e} \cdot P_\mathbb{ea} $$

\section{Билинейная форма}

$B$ - билинейная форма.

$B_\mathbb{e}$ - матрица $B$ в базисе $\mathbb{e}$.

$c$, $\lambda$ - некоторые числа.

$$ B(\boldsymbol{x}, \boldsymbol{y}) = c$$
$$ B(\boldsymbol{x_1} + \boldsymbol{x_2}, \boldsymbol{y}) = B(\boldsymbol{x_1}, \boldsymbol{y}) + B(\boldsymbol{x_2}, \boldsymbol{y}) $$
$$ B(\boldsymbol{x}, \boldsymbol{y_1} + \boldsymbol{y_2}) = B(\boldsymbol{x}, \boldsymbol{y_1}) + B(\boldsymbol{x}, \boldsymbol{y_2}) $$
$$ B(\lambda\cdot \boldsymbol{x}, \boldsymbol{y}) =  \lambda\cdot B(\boldsymbol{x}, \boldsymbol{y})$$
$$ B(\boldsymbol{x}, \lambda\cdot \boldsymbol{y}) =  \lambda\cdot B(\boldsymbol{x}, \boldsymbol{y})$$
$$ B(x, y) = B(y, x) \Leftrightarrow B_\mathbb{e} = B_\mathbb{e}^T $$
$$ c = [x]_\mathbb{e}^T \cdot B_\mathbb{e} \cdot [y]_\mathbb{e} $$
$$ B_\mathbb{a} = P_\mathbb{??}^T \cdot B_\mathbb{e} \cdot P_\mathbb{??} $$
$$ B_\mathbb{e} = \left\lceil B(\boldsymbol{e_i},\boldsymbol{e_j})\right\rfloor_{i,j=\overline{1,n}} $$

\subsection{Евклидовы пространства}

\subsection{Уравнения прямых/плоскостей}

\subsection{Кривые второго порядка}

\subsection{Жорданова форма и функции от матриц}

%--------------------------------------------------------------------------------%

\section{Геометрия}

\subsection{Планиметрия}

\subsection{Аксиомы}

\subsection{Теоремы}

\subsubsection{Треугольник}

\subsection{Окружность}

\subsection{Остальное}

\subsection{Правильный многоугольник}

\subsection{Стереометрия}